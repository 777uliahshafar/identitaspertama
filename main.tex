%%%%%%%%%%%%%%%%%%%%%%%%%%%%%%%%%%%%%%%%%
% Simple Article
% Integrated article template with simple for make4ht
% LaTeX Class
% Version 1.0 (10/11/20)
%
% This class originates by:
% Vel and  Nicolas Diaz
%
% Authors:
% Muhammad Uliah Shafar
%
%
% Free License:
%
%
%%%%%%%%%%%%%%%%%%%%%%%%%%%%%%%%%%%%%%%%%
\documentclass[11pt]{simart} % Font size (can be 10pt, 11pt or 12pt)

%----------------------------------------------------------------------------------------
%	TITLE SECTION
%----------------------------------------------------------------------------------------
% MAIN TITLE SECTION

\title{Studi Awal Elemen Fisik dalam Membentuk Identitas Perkotaan Tepi Laut: Kasus Tepi Laut Mattirotasi}



% Title and subtitle
%\date{\textbf{\DTMtoday}}
\date{\textbf{\today}}
\author{
\begin{tabular}{@{}ll@{}}
	Nama & : Muhammad Uliah Shafar\\
	NIM & : 21020119420029\\
\end{tabular}
}

%----------------------------------------------------------------------------------------
% OTHER TITLE SECTION

%\title{\textbf{Sistem Sarana dan Prasarana Jl. Pinggir Laut} \\ {\Large\itshape Infrastructure of Waterfront Parepare City}} % Title and subtitle

%\author{\textbf{Uliah Shafar} \\ \textit{Universitas Diponegoro}} % Author and institution

%\date{\today} % Date, use \date{} for no date

%----------------------------------------------------------------------------------------


% \AddToHook{cmd/section/before}{\clearpage} %start each section in new page

\begin{document}
\thispagestyle{empty}
\begin{center}
	\begin{huge}
		\bf{Lorem Ipsum Lorem Ipsum}
	\end{huge}

	\vspace{20pt}
	\includegraphics[width=0.35\textwidth]{logo} \\

	\vspace*{35pt}

	\begin{large}
		\textbf{Lorem Ipsum Lorem Ipsum} \\
		Lorem Ipsum Lorem\\

		\vspace{20pt}
		\textbf{oleh\\
			\vspace{20pt}
			Muhammad Uliah Shafar\\21020119420029}\\

		\vspace{20pt}
		Dosen: \\
		\textbf{Lorem Ipsum}\\



		\vspace{60pt}
		\textbf{PROGRAM STUDI MAGISTER ARSITEKTUR DEPARTEMEN ARSITEKTUR\\
			UNIVERSITAS DIPONEGORO\\
			SEMARANG\\
			2020
		}
	\end{large}
\end{center}
\clearpage

\maketitle % Print the title section

%----------------------------------------------------------------------------------------
%	ABSTRACT AND KEYWORDS
%----------------------------------------------------------------------------------------

\begin{abstract}
	Penelitian ini bertujuan untuk menganalisis identitas ruang publik di kawasan pesisir yang semakin berkembang, sehingga menimbulkan pertanyaan terkait identitasnya. Hasil penelitian ini diharapkan dapat memberikan kontribusi dalam merancang dan merencanakan ruang publik yang lebih efektif dan berkesan di masa depan. Sebagai bagian dari upaya untuk memahami identitas ruang publik, penelitian ini menggali lebih dalam tentang bagaimana bentuk fisik, aktivitas, dan makna saling berinteraksi untuk membentuk karakter unik dari ruang-ruang tersebut.
	Metode penelitian yang diterapkan melibatkan pendekatan kualitatif deskriptif. Penelitian berfokus pada studi kasus tunggal untuk mengumpulkan data tentang fenomena terkait bentuk fisik, aktivitas, dan makna. Selanjutnya analisis dilakukan secara deskriptif-kualitatif dengan menghubungkan  temuan lapangan dengan teori yang ada.
	Pengumpulan data dilakukan melalui observasi langsung, kemudian data dianalisis berdasarkan studi literatur. Kawasan yang dipilih sebagai studi kasus adalah tepi laut Mattirotasi.

	Hasil penelitian menunjukkan bahwa elemen-elemen fisik pada ruang publik ini belum cukup kuat dalam mempertahankan identitas ruang publik tepi laut ini. Hal ini disebabkan oleh kurangnya aksesibilitas, rendahnya keterbacaan, degradasi lingkungan perairan, dan keterhubungan ruang publik yang terbatas. Sebagai akibatnya, kualitas hubungan dengan lingkungan sekitar pun terganggu.
	Penelitian awal ini memberikan kontribusi penting dalam mengisi celah literatur mengenai identitas ruang publik di tepi laut, serta memberikan implikasi praktis bagi perancang kota dan pemangku kebijakan dalam merancang strategi pengelolaan ruang publik di daerah pesisir.
\end{abstract}

\hspace*{3.6mm}\textit{Kata Kunci:} Tepi laut Mattirotasi, Identitas Tempat, Elemen fisik, Ruang Publik % Keywords

---
\begin{abstract}
	This study aims to analyze the waterfront identity of Mattirotasi Street (Matras) in Parepare, focusing on specific aspects such as physical form, activities, and meanings. The results of this research are expected to contribute to designing and planning more effective and impactful public spaces in the future. As part of an effort to understand the identity of the waterfront, this paper delves deeper into how physical forms, activities, and meanings interact to form the unique character of these spaces. The research method applied involves a qualitative approach. The study focuses on a single case study to gather data on phenomena related to the physical form, activities, and meanings, which are then referred to as spatial aspects. Data collection was carried out through direct observation, and the data was analyzed based on literature studies. The area chosen as a case study is the coastline of Mattirotasi. The results of the study show a number of spatial aspects that have been eroded, leading to poor accessibility and legibility, a decline in the quality of the water environment, and a lack of visual and functional connectivity between public spaces and the surrounding areas. These aspects include accessibility, visual quality, comfort, uniqueness, legibility, strong sense of place attachment, degree of enclosure, primary activities, and attractiveness. These aspects are represented through physical design, which causes changes in the waterfront public space identity.
\end{abstract}

\hspace*{3.6mm}\textit{Keywords:} Waterfront, Place Identity, Mattirotasi, Identity Aspect % Keywords

\vspace{30pt} % Vertical whitespace between the abstract and first section

%----------------------------------------------------------------------------------------
%	ESSAY BODY
%----------------------------------------------------------------------------------------
% adding pdf in altex
% \includepdf[pages={1-2},pagecommand={\thispagestyle{plain}\fakesection{}}]{file.pdf}

\section{Pendahuluan}

Parepare sebagai kota yang memiliki tepi laut belakangan ini mengalami pertumbuhan penduduk yang signifikan \citep{rusman2020}. Menurut data BPS, penduduk Parepare saat ini adalah 163.314 orang.
Seiring dengan pertambahan populasi, permintaan terhadap ruang publik juga mengalami peningkatan. Hal ini tercermin dari  pertumbuhan sektor bisnis dan ekonomi. Menurut \cite{breen1994}, permintaan tersebut berdasarkan kecenderungan masyarkat dalam mencari tempat rekreasi \textit{(restorative)}.

Ruang publik memiliki beragam fungsi, seperti rekreasi, pekerjaan, bisnis, kegiatan sosial budaya, dan warisan budaya \citep{hajmirsadeghi2012}. Contoh bentuk ruang publik meliputi taman, alun-alun, tepi laut, dan jalan \citep{hajmirsadeghi2012}. Tempat ini adalah tempat di mana individu atau kelompok melakukan berbagai aktivitas untuk mencari kesenangan dan manfaat positif lainnya \citep{hajmirsadeghi2012}.
Ruang publik juga merupakan bagian terpenting dari suatu perkotaan \citep{dong2004}. Masyarakat semakin menyadari pentingnya ruang publik ketika mereka mencari tempat yang mudah diakses untuk melakukan aktivitas sehari-hari. Mereka menggunakan ruang publik sebagai tempat untuk melakukan berbagai kegiatan rutin, seperti menghabiskan waktu di jalan yang sering dilalui atau bermain di alun-alun bersama keluarga.

Ruang publik yang berkualitas memiliki identitas yang kuat \citep{hartanti2014}. Identitas, menurut \cite{hartanti2014}, merupakan landasan untuk mengenali sesuatu sebagai entitas yang berdiri sendiri. Kamus Webster’s Ninth New Collegiate mendefinisikan identitas sebagai karakteristik atau kondisi yang membedakan sesuatu. Identitas suatu tempat dapat terbentuk melalui berbagai karakteristik seperti bentuk fisik, aktivitas, aspek sosial, dan makna yang terkandung di dalamnya. Identitas ruang publik secara fisik merujuk pada karakteristik visual dan strukturalnya. Sedangkan secara sosial, identitas merujuk pada





Dalam konteks penelitian ini, teori identitas tempat akan digunakan sebagai kerangka kerja untuk menganalisis bagaimana karakteristik ruang publik di Parepare mencerminkan identitas tempat tersebut, serta bagaimana identitas ini dapat memengaruhi persepsi dan pemanfaatan ruang publik oleh masyarakat lokal.




Identitas yang kuat akan mendukung ruang publik menjadi lebih baik dengan berbagai cara, diantaranya: memperkokoh perbedaannya, menciptakan kegiatan atau penanda khusus, dan mendukung sejumlah motif tertentu \citep{hartanti2014}. Seperti contoh, festival Salo Karajae merupakan aktivitas tahunan di Tonrangeng Riverside. Festival ini membedakan ruang pesisir laut Tonrangeng diantara ruang publik lainnya yang ada di kota Parepare. Apabila identitas suatu ruang publik kuat, maka orang akan tertarik untuk datang dan singgah untuk sementara waktu \citep{oktay2002}.

Parepare memiliki sejumlah ruang publik pada tepi laut Mattirotasi. Beberapa ruang publik tersebut baru saja selesai dibangun ataupun direnovasi. Pengembangan ini jelas mengubah tatanan sosial, budaya, morfologi suatu komposisi ruang pada tingkat kawasan maupun kota \citep{kaymaz2013,oktay2002}. Sehingga memunculkan pertanyaan terkait pertahanan identitas, dan kualitas lokal \citep{kaymaz2013}.

\cite{oktay2015} menyebutkan bahwa penelitian mengenai identitas ruang publik telah banyak dilakukan, namun hanya sedikit yang melakukannya pada ruang publik kawasan pesisir laut. Ruang publik ini lebih bernilai karena memiliki keanekaragaman karakter yang ditimbulkan oleh letaknya yang berdekatan dengan laut. Selain itu, memiliki kontribusi besar terhadap pengembangan sebuah kota \citep{hussein2014}. \cite{hussein2014} menekankan bahwa ruang publik pesisir laut yang berhasil mampu membawa masyarakat perkotaan ke pesisir laut. Salah satu caranya adalah memperkuat identitas ruang-ruang publik tersebut \citep{oktay2002}.

Penelitian ini bertujuan untuk menganalisis identitas ruang publik di kawasan tepi laut kota Parepare, khususnya di sepanjang Jalan Mattirotasi (Matras), dalam menghadapi penurunan identitas perkotaan akibat pertumbuhan penduduk. Fokus utamanya adalah memahami bagaimana bentuk fisik, aktivitas, dan makna di tepi laut Matras membentuk identitas kota Parepare, serta dampaknya terhadap persepsi dan pemanfaatan ruang publik oleh masyarakat lokal. Penelitian ini diharapkan dapat memberikan kontribusi dalam bidang arsitektur dan perencanaan perkotaan, terutama dalam konteks ruang publik di kawasan tepi laut, serta memberikan rekomendasi untuk pengembangan dan peningkatan kualitas ruang publik di kota-kota pesisir lain yang menghadapi tantangan serupa.

% \subfile{subfiles/subfile.tex}

\section{Metodologi Penelitian}
Untuk menganalisis identitas ruang publik pesisir laut kota Parepare, penelitian ini menggunakan pendekatan kualitatif. Metode penelitian kualitatif memberikan pemahaman mendalam tentang suatu fenomena melalui observasi langsung dan analisis mendalam terhadap objek penelitian, yang sering kali dikenal sebagai pendekatan studi kasus tunggal \citep{creswell2016}. Pendekatan studi kasus penelitian ini mengumpulkan data fenomena terkait bentuk fisik, aktivitas, dan makna.

Data dikumpulkan menggunakan tiga teknik yang berbeda, di antaranya observasi lapangan dilakukan untuk mendapatkan data primer, sementara kajian pustaka dilakukan untuk mendapatkan data sekunder. Selanjutnya, kajian dokumen dilakukan untuk mengevaluasi kebijakan dan ketentuan yang berlaku \citep{iqbal2020}. \cite{wiraguna2024} menjelaskan bahwa analisis data yang cermat merupakan bagian dari studi kasus. Oleh karena itu, penelitian ini selanjutnya menganalisis data yang telah tersedia dengan menggunakan metode analisis konten.

Lokasi studi kasus penelitian ini adalah tepi laut Mattirotasi. Lokasi ini memiliki sejumlah fasilitas umum seperti masjid terapung, foodcourt, taman, dan pedagang kaki lima. Selain itu, keberadaan rumah penduduk tua juga terhitung sebagai bentuk fisik yang berbeda.

\section{Hasil dan Pembahasan}%

Pada awal abad 21, Parepare merupakan salah satu kota destinasi yang sangat populer. Alasannya adalah kota ini memiliki lokasi yang strategis. Lokasi tersebut memungkinkan sejumlah pelancong untuk transit di kota ini. Selain itu, Parepare memiliki daya tarik tersendiri karena berhubungan langsung dengan laut dan daratan tinggi. Akibatnya, kota ini fokus dan berusaha keras untuk mengembangkan sektor pariwisata \citep{faniapriani2018}.

Salah satu upaya Parepare dalam meningkatkan sektor pariwisata adalah melalui pengembangan tepi laut (waterfront) Mattirotasi (Matras). Tepi laut Mattirotasi Parepare ini terletak di tengah kota dengan pemukiman padat penduduk. Banyak penduduk yang menghabiskan waktu luangnya di sini dengan menikmati keindahan laut, berjalan-jalan di sepanjang tepi laut, atau hanya duduk sambil menikmati angin laut. Pengembangan tepi laut Mattirotasi meliputi taman, trotoar, food court, tempat PKL (pedagang kaki lima), dan masjid terapung.

\subsection{Evaluasi Bentuk Fisik}%
\label{sub:Evaluasi Bentuk Fisik}

Bentuk fisik pada tepi laut Matras adalah bangunan dan lanskap, seperti taman, trotoar, jalan, pantai, tempat PKL, rumah, dan akomodasi lainnya. Namun, jalan yang berada di tepi laut ini tidak menunjukkan aspek aksesibilitas yang baik, terutama disebabkan oleh adanya penghalang berupa tembok dan perbedaan elevasi yang signifikan. \cite{iqbal2020} menjelaskan bahwa kondisi tersebut merupakan kekurangan dalam ruang publik yang mempengaruhi legibilitas dan melemahkan identitas tepi laut.

Jalan di sepanjang tepi laut juga perlu memperhatikan aspek permeabilitas \citep{wanismail2018}. Aspek ini memberikan orang pilihan untuk memilih rute yang dilalui. Jalan yang berada di sepanjang garis pantai hanya dapat dilalui dengan berjalan kaki karena berbagai faktor seperti perbedaan elevasi dengan jalan raya, ketersediaan jalan pedestrian yang terhubung, jalan pintas, dan trotoar yang aman. Meskipun jalan tidak menyediakan akses ke garis pantai, jalan yang terintegrasi ini dapat mendorong aksesibilitas dalam ruang publik.

Keterlihatan memiliki peran yang sama pentingnya dengan aksesibilitas terhadap garis pantai. \cite{wanismail2018} menjelaskan bahwa pemandangan dapat terhalang oleh dua komponen yaitu bangunan (misalnya lokasi bangunan, tipe, tinggi, desain fasad) dan lanskap (air, batako jalan, perabot jalan). Berdasarkan analisis, kehadiran masjid terapung, rumah penduduk, dan keragaman pada rancangan lanskap meningkatkan keterlihatan. Masjid terapung menawarkan desain yang unik dengan bentuk bundar. Sementara itu, lanskap memiliki rancangan yang modern disertai sejumlah street furniture (dekorasi jalan) yang beragam.

\subsection{Evaluasi Aktivitas yang diamati}%
\label{sub:Evaluasi Aktivitas yang diamati}

Berbagai aktivitas yang dilakukan oleh individu yang berbeda dapat meningkatkan kualitas tepi laut. Aktivitas atau suatu kegiatan melibatkan bentuk fisik sebuah ruang publik yang mempengaruhi tatanan fisik \citep{wanismail2018}. Aktivitas yang dominan di tepi laut Matras adalah menikmati suasana laut dengan berbagai cara, seperti melihat pemandangan, duduk santai, dan berjalan-jalan sepanjang tepi laut. Pada malam hari, aktivitas ritel juga terlaksana di tepi laut, menambah keragaman pada tatanan fisik ruang publik. \cite{wanismail2018} menyebutkan bahwa aktivitas tersebut juga menciptakan suasana yang menarik dan hidup. Beragam aktivitas ini membantu tepi laut untuk mempertahankan keramaian dan memperkuat identitas tepi laut \citep{iqbal2020}.

\subsection{Evaluasi Makna dan Relevansi}%
\label{sub:Evaluasi Makna dan Relevansi}


Rumah-rumah bersejarah yang masih berjejeran di sekitar tepi laut ini meningkatkan aspek simbolisme dan memori tempat tersebut. Menurut \cite{iqbal2020}, rumah tersebut memiliki keunikan dalam bentuk fisik, budaya, dan sejarahnya. Namun, beberapa rumah telah diubah menjadi warung makanan dan jarak antara titik kumpul dan bangunan jauh, mengurangi karakteristik bangunan bersejarah. Orang yang seharusnya menikmati suasana tepi laut bersejarah mengalami kesulitan.


Baru-baru ini, pengembangan hanya melibatkan sedikit pembangunan fasilitas dan peningkatan area hijau. Meskipun demikian, sejumlah pengguna merasa nyaman dengan tepi laut Matras dan sekitarnya karena kebanyakan dari mereka adalah orang lokal yang akrab dengan tempat tersebut. Hal ini terlihat dari lamanya mereka menggunakan ruang tepi laut ini. \cite{ujang2017} menyebutkan bahwa keakraban dapat mempengaruhi perasaan nyaman individu dalam tatanan fisik. Kondisi ini tampak khususnya pada malam hari di mana tingkat keramaiannya sedang hingga tinggi.


\section{Kesimpulan}%
\label{sec:Kesimpulan}

Tepi laut Matras dianalisis untuk menentukan bentuk fisiknya yang penting yang telah berkontribusi pada aspek-aspek ruang publik. Penelitian ini menyimpulkan terdapat sejumlah aspek yang perlu ditingkatkan dan dipertahankan pada setiap karakter tepi laut (bentuk fisik, aktivitas, dan makna) dalam memperkuat identitas ruang publik. Pada karakter bentuk fisik, aksesibilitas dan keterlihatan terhadap garis pantai dapat mengurangi kualitas identitas tepi laut. Masih banyak bentuk fisik yang tidak mendukung aspek aksesibilitas dan keterlihatan. Dalam hal aktivitas, ritel menjadi faktor pendorong utama dan memperkuat aktivitas yang terjadi di sekitar garis pantai. Pada karakter makna dan relevansi, rumah tua yang berjejeran sepanjang jalan raya tepi laut memberikan makna dan simbolisme. Namun, sedikit demi sedikit tempat ini beralih fungsi menjadi warung makanan. Setiap aspek mendukung karakter tempat dalam memperkuat identitas tepi laut yang unik. Aspek ini membantu memberikan pemahaman yang lebih baik tentang identitas tepi laut secara umum kepada pemangku kebijakan dan perancang perkotaan. Adapun pembahasan pada penelitian ini masih berdasarkan pada studi pilot, sehingga perlu dilakukan analisis yang lebih mendalam.




%----------------------------------------------------------------------------------------
%	BIBLIOGRAPHY
%----------------------------------------------------------------------------------------
% Comment these out if they are already in sub
\bibliographystyle{apalike}

% \bibliography{biblio.bib}
\bibliography{urbanidentity.bib}


\end{document}
