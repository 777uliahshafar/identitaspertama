%%%%%%%%%%%%%%%%%%%%%%%%%%%%%%%%%%%%%%%%%
% Simple Article
% Integrated article template with simple for make4ht
% LaTeX Class
% Version 1.0 (10/11/20)
%
% This class originates by:
% Vel and  Nicolas Diaz
%
% Authors:
% Muhammad Uliah Shafar
%
%
% Free License:
%
%
%%%%%%%%%%%%%%%%%%%%%%%%%%%%%%%%%%%%%%%%%
\documentclass[11pt]{simart} % Font size (can be 10pt, 11pt or 12pt)

%----------------------------------------------------------------------------------------
%	TITLE SECTION
%----------------------------------------------------------------------------------------
% MAIN TITLE SECTION

\title{Studi Awal Aspek dalam Membentuk Identitas Perkotaan Tepi Laut: Kasus Tepi Laut Mattirotasi}



% Title and subtitle
%\date{\textbf{\DTMtoday}}
\date{\textbf{\today}}
\author{
\begin{tabular}{@{}ll@{}}
	Nama & : Muhammad Uliah Shafar\\
	NIM & : 21020119420029\\
\end{tabular}
}

%----------------------------------------------------------------------------------------
% OTHER TITLE SECTION

%\title{\textbf{Sistem Sarana dan Prasarana Jl. Pinggir Laut} \\ {\Large\itshape Infrastructure of Waterfront Parepare City}} % Title and subtitle

%\author{\textbf{Uliah Shafar} \\ \textit{Universitas Diponegoro}} % Author and institution

%\date{\today} % Date, use \date{} for no date

%----------------------------------------------------------------------------------------


% \AddToHook{cmd/section/before}{\clearpage} %start each section in new page

\begin{document}
\thispagestyle{empty}
\begin{center}
	\begin{huge}
		\bf{Lorem Ipsum Lorem Ipsum}
	\end{huge}

	\vspace{20pt}
	\includegraphics[width=0.35\textwidth]{logo} \\

	\vspace*{35pt}

	\begin{large}
		\textbf{Lorem Ipsum Lorem Ipsum} \\
		Lorem Ipsum Lorem\\

		\vspace{20pt}
		\textbf{oleh\\
			\vspace{20pt}
			Muhammad Uliah Shafar\\21020119420029}\\

		\vspace{20pt}
		Dosen: \\
		\textbf{Lorem Ipsum}\\



		\vspace{60pt}
		\textbf{PROGRAM STUDI MAGISTER ARSITEKTUR DEPARTEMEN ARSITEKTUR\\
			UNIVERSITAS DIPONEGORO\\
			SEMARANG\\
			2020
		}
	\end{large}
\end{center}
\clearpage

\maketitle % Print the title section

%----------------------------------------------------------------------------------------
%	ABSTRACT AND KEYWORDS
%----------------------------------------------------------------------------------------

\begin{abstract}
	Penelitian ini bertujuan untuk menganalisis identitas ruang publik di kawasan pesisir yang semakin berkembang, sehingga menimbulkan pertanyaan terkait identitasnya. Hasil penelitian ini diharapkan dapat memberikan kontribusi dalam merancang dan merencanakan ruang publik yang lebih efektif dan berkesan di masa depan. Sebagai bagian dari upaya untuk memahami identitas ruang publik, penelitian ini menggali lebih dalam tentang bagaimana bentuk fisik, aktivitas, dan makna saling berinteraksi untuk membentuk karakter unik dari ruang-ruang tersebut.
	Metode penelitian yang diterapkan melibatkan pendekatan kualitatif deskriptif. Penelitian berfokus pada studi kasus tunggal untuk mengumpulkan data tentang fenomena terkait bentuk fisik, aktivitas, dan makna. Selanjutnya analisis dilakukan secara deskriptif-kualitatif dengan menghubungkan  temuan lapangan dengan teori yang ada.
	Pengumpulan data dilakukan melalui observasi langsung, kemudian data dianalisis berdasarkan studi literatur. Kawasan yang dipilih sebagai studi kasus adalah tepi laut Mattirotasi.

	Hasil penelitian menunjukkan bahwa aspek-aspek pada ruang publik ini belum cukup kuat dalam mempertahankan identitas ruang publik tepi laut ini. Hal ini disebabkan oleh kurangnya aksesibilitas, rendahnya keterbacaan, degradasi lingkungan perairan, dan keterhubungan ruang publik yang terbatas. Sebagai akibatnya, kualitas hubungan dengan lingkungan sekitar pun terganggu.
	Penelitian awal ini memberikan kontribusi penting dalam mengisi celah literatur mengenai identitas ruang publik di tepi laut, serta memberikan implikasi praktis bagi perancang kota dan pemangku kebijakan dalam merancang strategi pengelolaan ruang publik di daerah pesisir.
\end{abstract}

\hspace*{3.6mm}\textit{Kata Kunci:} Tepi laut Mattirotasi, Identitas Tempat, Aspek-aspek ruang publik, Ruang Publik % Keywords

---
\begin{abstract}
	This study aims to analyze the waterfront identity of Mattirotasi Street (Matras) in Parepare, focusing on specific aspects such as physical form, activities, and meanings. The results of this research are expected to contribute to designing and planning more effective and impactful public spaces in the future. As part of an effort to understand the identity of the waterfront, this paper delves deeper into how physical forms, activities, and meanings interact to form the unique character of these spaces. The research method applied involves a qualitative approach. The study focuses on a single case study to gather data on phenomena related to the physical form, activities, and meanings, which are then referred to as spatial aspects. Data collection was carried out through direct observation, and the data was analyzed based on literature studies. The area chosen as a case study is the coastline of Mattirotasi. The results of the study show a number of spatial aspects that have been eroded, leading to poor accessibility and legibility, a decline in the quality of the water environment, and a lack of visual and functional connectivity between public spaces and the surrounding areas. These aspects include accessibility, visual quality, comfort, uniqueness, legibility, strong sense of place attachment, degree of enclosure, primary activities, and attractiveness. These aspects are represented through physical design, which causes changes in the waterfront public space identity.
\end{abstract}

\hspace*{3.6mm}\textit{Keywords:} Waterfront, Place Identity, Mattirotasi, Identity Aspect % Keywords

\vspace{30pt} % Vertical whitespace between the abstract and first section

%----------------------------------------------------------------------------------------
%	ESSAY BODY
%----------------------------------------------------------------------------------------
% adding pdf in altex
% \includepdf[pages={1-2},pagecommand={\thispagestyle{plain}\fakesection{}}]{file.pdf}

\section{Pendahuluan}

Parepare sebagai kota yang memiliki tepi laut belakangan ini mengalami pertumbuhan penduduk yang signifikan \citep{rusman2020}. Menurut data BPS, penduduk Parepare saat ini adalah 163.314 orang.
Seiring dengan pertambahan populasi, permintaan terhadap ruang publik juga mengalami peningkatan. Hal ini tercermin dari  pertumbuhan sektor bisnis dan ekonomi. Menurut \cite{breen1994}, permintaan tersebut berdasarkan kecenderungan masyarkat dalam mencari tempat rekreasi \textit{(restorative)}.

Ruang publik memiliki beragam fungsi, seperti rekreasi, pekerjaan, bisnis, kegiatan sosial budaya, dan warisan budaya \citep{hajmirsadeghi2012}. Contoh bentuk ruang publik meliputi taman, alun-alun, tepi laut, dan jalan \citep{hajmirsadeghi2012}. Tempat ini adalah tempat di mana individu atau kelompok melakukan berbagai aktivitas untuk mencari kesenangan dan manfaat positif lainnya \citep{hajmirsadeghi2012}.
Ruang publik juga merupakan bagian terpenting dari suatu perkotaan \citep{dong2004}. Masyarakat semakin menyadari pentingnya ruang publik ketika mereka mencari tempat yang mudah diakses untuk melakukan aktivitas sehari-hari. Mereka menggunakan ruang publik sebagai tempat untuk melakukan berbagai kegiatan rutin, seperti menghabiskan waktu di jalan yang sering dilalui atau bermain di alun-alun bersama keluarga.

Ruang publik yang berkualitas memiliki identitas yang kuat \citep{hartanti2014}.
Identitas yang kuat memiliki peran penting sebagai penanda yang khusus, inisiator kegiatan, dan penguat perbedaan \citep{hartanti2014}. Selain itu, identitas juga berperan meningkatkan pengunjung dalam suatu tempat \citep{oktay2002}. Identitas, menurut \cite{hartanti2014}, merupakan landasan untuk mengenali sesuatu sebagai entitas yang berdiri sendiri. Kamus Webster’s Ninth New Collegiate mendefinisikan identitas sebagai karakteristik atau kondisi yang membedakan sesuatu. Identitas suatu tempat dapat terbentuk melalui berbagai karakteristik seperti bentuk fisik, aktivitas, aspek sosial, dan makna yang terkandung di dalamnya.

Identitas ruang publik secara fisik merujuk pada karakteristik visual dan strukturalnya.
Sedangkan secara sosial, identitas merujuk pada kegiatan aktif yang terjadi pada ruang publik \citep{al2023}. Sebagai contoh, permainan skateboard yang terjadi secara periodik. Berbeda dengan identitas secara makna dimana merujuk pada tujuan untuk mencerminkan atau memperkuat sesuatu \citep{sari2024}.
Dalam konteks penelitian ini, teori identitas tempat akan digunakan sebagai kerangka kerja untuk menganalisis bagaimana aspek-aspek ruang publik di Parepare mencerminkan identitas tempat tersebut, serta bagaimana identitas ini dapat memengaruhi persepsi dan pemanfaatan ruang publik oleh masyarakat lokal.

Kota Parepare menawarkan berbagai fasilitas ruang publik yang terletak di sepanjang kawasan tepi laut Pantai Mattirotasi. Sejumlah area publik ini baru-baru ini telah mengalami proses pembangunan atau renovasi. Menurut \cite{kaymaz2013}, pembangunan tersebut memicu pertanyaan mengenai ketahanan identitas dan lokalitas dari area-area publik tersebut. Oleh karena itu, penelitian terkait identitas pada daerah yang berkembang adalah sangat penting.

Penelitian terkait identitas telah banyak dilakukan \citep{oktay2015}. Namun hanya sedikit yang melakukannya pada ruang publik kawasan pesisir laut. Penelitian banyak dilakukan pada sudut-sudut perkotaan seperti jalan trotoar \citep{wanismail2018}.
Dengan demikian, studi yang lebih komprehensif di kawasan tepi menjadi krusial untuk memperdalam pemahaman tentang dinamika identitas setempat.

\begin{comment}
hanya sedikit yang melakukannya pada ruang publik kawasan pesisir laut.
Ruang publik ini lebih bernilai karena memiliki keanekaragaman karakter yang ditimbulkan oleh letaknya yang berdekatan dengan laut.
Selain itu, memiliki kontribusi besar terhadap pengembangan sebuah kota \citep{hussein2014}. \cite{hussein2014} menekankan bahwa ruang publik pesisir laut yang berhasil mampu membawa masyarakat perkotaan ke pesisir laut. Salah satu caranya adalah memperkuat identitas ruang-ruang publik tersebut \citep{oktay2002}.
\end{comment}

Penelitian ini bertujuan untuk menganalisis identitas ruang publik di kawasan tepi laut Kota Parepare, khususnya di sepanjang Jalan Mattirotasi (Matras), dalam konteks penurunan identitas perkotaan akibat pertumbuhan penduduk.

Tujuan dari penelitian ini adalah untuk menganalisis identitas ruang publik di sepanjang Jalan Mattirotasi (Matras) di Kota Parepare yang berada di tepi laut.
Fokus penelitian ini adalah memahami peran bentuk fisik, aktivitas, dan makna di tepi laut Matras dalam membentuk identitas kota Parepare, serta dampaknya terhadap persepsi dan pemanfaatan ruang publik oleh masyarakat lokal.
Hasil penelitian ini diharapkan dapat berkontribusi pada bidang arsitektur dan perencanaan perkotaan, terutama dalam konteks ruang publik di daerah tepi laut.
Selain itu, penelitian ini juga diharapkan dapat memberikan rekomendasi yang berguna untuk meningkatkan kualitas ruang publik di kota-kota tepi laut lainnya yang menghadapi masalah serupa.


% \subfile{subfiles/subfile.tex}

\section{Metodologi Penelitian}
% Untuk menganalisis identitas ruang publik pesisir di Parepare, penelitian ini menggunakan pendekatan kualitatif dengan desain studi kasus tunggal karena fenomena yang ditelaah bersifat kontekstual dan kompleks, sehingga memerlukan *thick description* atas interaksi antara bentuk fisik, aktivitas, dan makna. Desain ini memungkinkan penelusuran mekanisme—bukan sekadar gejala—melalui observasi lapangan dan telaah dokumen, sehingga menghasilkan penjelasan yang kaya serta terikat konteks mengenai koridor tepi laut Mattirotasi \citep{creswell2016}.
Desain ini memungkinkan eksplorasi mendalam terhadap proses yang mendasari fenomena tersebut—bukan sekadar gejalanya—melalui observasi lapangan dan telaah dokumen.


Untuk menganalisis identitas ruang publik pesisir di Parepare, penelitian ini menggunakan pendekatan kualitatif dengan desain studi kasus tunggal karena fenomena yang ditelaah bersifat kontekstual dan kompleks, sehingga memerlukan deskripsi mendalam atas interaksi antara bentuk fisik, aktivitas, dan makna \citep{creswell2016}.
Desain ini memungkinkan penelusuran mendalam melalui observasi lapangan dan analisis dokumen, bukan hanya mengidentifikasi gejala. Dengan demikian, penelitian ini mampu menjelaskan identitas ruang publik tepi laut Kota Parepare.

\begin{comment}
sehingga menghasilkan penjelasan yang kaya serta terikat konteks mengenai koridor tepi laut Mattirotasi \citep{creswell2016}.


Untuk menganalisis identitas ruang publik pesisir laut kota Parepare, penelitian ini menggunakan pendekatan kualitatif. Metode penelitian kualitatif memberikan pemahaman mendalam tentang suatu fenomena melalui observasi langsung dan analisis mendalam terhadap objek penelitian, yang sering kali dikenal sebagai pendekatan studi kasus tunggal \citep{creswell2016}. Pendekatan studi kasus penelitian ini mengumpulkan data fenomena terkait bentuk fisik, aktivitas, dan makna.
\end{comment}

% Untuk menganalisis identitas ruang publik pesisir laut kota Parepare, penelitian ini menggunakan pendekatan kualitatif yang memungkinkan pemahaman mendalam tentang fenomena yang diteliti. Metode ini dipilih karena kemampuannya untuk menggali informasi secara mendetail melalui observasi langsung dan analisis mendalam terhadap objek penelitian. Dengan menggunakan pendekatan studi kasus tunggal, penelitian ini mengumpulkan data terkait bentuk fisik, aktivitas, dan makna yang ada di ruang publik tersebut \citep{creswell2016}. Pendekatan ini memberikan wawasan yang lebih kaya dan kontekstual, sehingga dapat mengungkapkan dinamika dan karakteristik unik dari ruang publik pesisir laut kota Parepare.

% pengumpulan data

Data dikumpulkan melalui tiga teknik yang berbeda, yaitu survei lapangan, wawancara, dan kajian literatur. Survei lapangan dilakukan untuk mendapatkan data primer yang mendalam, sementara wawancara memberikan wawasan langsung dari para informan terkait. Kajian literatur dilakukan untuk mendapatkan data sekunder dan mengevaluasi kebijakan serta ketentuan yang berlaku. Penelitian ini dilakukan di tepi laut jalan mattirotasi, dengan objek ruang publik sekitar taman terapung dan pantaiku.
Pengumpulan data dirumuskan tidak hanya untuk mendokumentasikan fenomena ruang publik, tetapi juga untuk memastikan bahwa interaksi antara bentuk fisik, sosial, dan makna dapat dievaluasi secara komprehensif dalam kerangka analisis identitas ruang tepi laut.

\cite{wiraguna2024}  menjelaskan bahwa menganalisis data secara teliti merupakan salah satu komponen penting dalam sebuah studi kasus.
Penelitian ini kemudian memanfaatkan metode analisis konten untuk mengolah data yang telah dikumpulkan. Penelitian ini menganalisis bagaimana ketiga aspek tersebut saling berinteraksi untuk membentuk identitas dari ruang publik tepi laut.
Analisis konten dalam penelitian ini difokuskan pada tiga aspek utama yang membentuk identitas ruang publik tepi laut.
Pertama, elemen fisik yang melibatkan analisis terhadap elemen visual dan struktural dari ruang publik.
Kedua, aspek aktivitas yang menitikberatkan pada identifikasi serta interpretasi berbagai kegiatan aktif masyarakat yang berlangsung di dalamnya.
Ketiga, aspek makna yang menyoroti simbolisme dan memori kolektif yang melekat pada ruang publik tersebut.
Ketiga dimensi ini dianalisis secara terpadu untuk memperoleh pemahaman yang komprehensif mengenai bagaimana ruang publik di kawasan tepi laut membangun dan mempertahankan identitasnya \citep{iqbal2020,ujang2017}.

---
Lokasi studi kasus penelitian ini adalah tepi laut Mattirotasi. Lokasi ini memiliki sejumlah fasilitas umum seperti masjid terapung, foodcourt, taman, dan pedagang kaki lima. Selain itu, keberadaan rumah penduduk tua juga terhitung sebagai bentuk fisik yang berbeda.

Lokasi studi kasus penelitian ini adalah tepi laut Mattirotasi, yang dipilih karena... [alasan metodologis utama, misalnya, relevansi dengan topik penelitian]. Lokasi ini memiliki sejumlah fasilitas umum seperti masjid terapung, foodcourt, taman, dan pedagang kaki lima, yang... [alasan spesifik terkait fasilitas, misalnya, mencerminkan interaksi sosial dan budaya]. Selain itu, keberadaan rumah penduduk tua juga terhitung sebagai bentuk fisik yang berbeda, yang... [alasan terkait elemen historis atau kultural, misalnya, menambah dimensi historis dan kultural untuk analisis].

Lokasi studi kasus penelitian ini adalah tepi laut Mattirotasi, yang dipilih karena memiliki keanekaragaman karakter yang ditimbulkan oleh letaknya yang berdekatan dengan laut dan memiliki kontribusi besar terhadap pengembangan sebuah kota.
Sejak pergantian kepemimpinan walikota Parepare, terjadi perkembangan signifikan di lokasi tersebut dengan adanya Pantai baru yang dikenal sebagai Pantaiku.
Lokasi ini memiliki sejumlah fasilitas umum seperti masjid terapung, foodcourt, taman, dan pedagang kaki lima, yang menciptakan beragam aktivitas dan membantu tepi laut untuk mempertahankan keramaian serta memperkuat identitas tepi laut.
Selain itu, rumah-rumah penduduk tua juga berkaitan dengan makna tempat.

% komentar buat maps lokasi

\section{Hasil dan Pembahasan}%

Pada awal abad 21, Parepare merupakan salah satu kota destinasi yang sangat populer. Alasannya adalah kota ini memiliki lokasi yang strategis. Lokasi tersebut memungkinkan sejumlah pelancong untuk transit di kota ini. Selain itu, Parepare memiliki daya tarik tersendiri karena berhubungan langsung dengan laut dan daratan tinggi. Akibatnya, kota ini fokus dan berusaha keras untuk mengembangkan sektor pariwisata \citep{faniapriani2018}.

Salah satu upaya Parepare dalam meningkatkan sektor pariwisata adalah melalui pengembangan tepi laut (waterfront) Mattirotasi (Matras). Tepi laut Mattirotasi Parepare ini terletak di tengah kota dengan pemukiman padat penduduk. Banyak penduduk yang menghabiskan waktu luangnya di sini dengan menikmati keindahan laut, berjalan-jalan di sepanjang tepi laut, atau hanya duduk sambil menikmati angin laut. Pengembangan tepi laut Mattirotasi meliputi taman, trotoar, food court, tempat PKL (pedagang kaki lima), dan masjid terapung.

\subsection{Evaluasi Bentuk Fisik}%
\label{sub:Evaluasi Bentuk Fisik}

Bentuk fisik pada tepi laut Matras adalah bangunan dan lanskap, seperti taman, trotoar, jalan, pantai, usaha mikro-menengah, rumah, dan akomodasi lainnya.
Jalan-jalan pada tepi laut ini tidak terhubung ke garis pantai.
Desain jalan ini bahkan menggunakan penghalang berupa tembok yang biasa digunakan untuk duduk santai.
\cite{iqbal2020} menjelaskan bahwa kondisi tersebut merupakan kekurangan dalam ruang publik yang mempengaruhi legibilitas dan melemahkan identitas tepi laut.
Oleh karena itu, diperlukan peningkatan yang signifikan pada elemen fisik.

Selain itu, jalan di sepanjang tepi laut juga perlu memperhatikan aspek permeabilitas \citep{wanismail2018}, untuk memastikan aksesibilitas yang lebih baik dan mendukung aliran pergerakan yang lebih lancar di area tersebut.
Jalan yang mengikuti garis pantai hanya dapat diakses dengan berjalan kaki karena faktor-faktor seperti elevasi yang berbeda dengan jalan raya, ketersediaan jalan pedestrian yang terhubung, jalan pintas, dan keberadaan trotoar yang aman.
Permeabilitas jalan yang tinggi akan meningkatkan aksesibilitas meskipun tidak terhubung ke garis pantai.

Keterlihatan dan aksesibilitas garis pantai memiliki peran yang setara dalam menentukan fungsionalitas dan daya tarik kawasan tepi laut.
\cite{wanismail2018} menjelaskan bahwa pemandangan dapat terhalang oleh dua komponen yaitu bangunan (misalnya lokasi bangunan, tipe, tinggi, desain fasad) dan lanskap (air, batako jalan, perabot jalan). Berdasarkan analisis, kehadiran masjid terapung, rumah penduduk, dan keragaman pada rancangan lanskap meningkatkan keterlihatan. Masjid terapung menawarkan desain yang unik dengan bentuk bundar. Sementara itu, lanskap memiliki rancangan yang modern disertai sejumlah street furniture (dekorasi jalan) yang beragam.

\subsection{Evaluasi Aktivitas yang diamati}%
\label{sub:Evaluasi Aktivitas yang diamati}

Berbagai aktivitas yang dilakukan oleh individu yang berbeda dapat meningkatkan kualitas tepi laut. Aktivitas atau suatu kegiatan melibatkan bentuk fisik sebuah ruang publik yang mempengaruhi tatanan fisik \citep{wanismail2018}. Aktivitas yang dominan di tepi laut Matras adalah menikmati suasana laut dengan berbagai cara, seperti melihat pemandangan, duduk santai, dan berjalan-jalan sepanjang tepi laut. Pada malam hari, aktivitas ritel juga terlaksana di tepi laut, menambah keragaman pada tatanan fisik ruang publik. \cite{wanismail2018} menyebutkan bahwa aktivitas tersebut juga menciptakan suasana yang menarik dan hidup. Beragam aktivitas ini membantu tepi laut untuk mempertahankan keramaian dan memperkuat identitas tepi laut \citep{iqbal2020}.

\subsection{Evaluasi Makna dan Relevansi}%
\label{sub:Evaluasi Makna dan Relevansi}

Elemen fisik yang ada di suatu tempat dapat menjadi simbol atau makna yang memperkuat identitas tempat \citep{hull1994}. Hunter dalam \cite{hull1994}  juga menyatakan bahwa orang mencari kehidupan yang menggabungkan antara makna dan identitas pribadi, sehingga elemen fisik yang ada di sekitar mereka dapat memainkan peran penting dalam proses tersebut.
Sebagai contoh, sebuah rumah tua yang asli dan indah dapat menjadi simbol sejarah bagi masyarakat di sekitarnya, sehingga tidak mengherankan bahwa topik rumah adalah paling sering diteliti \citep{hull1994}.

Tepi laut Mattirotasi dikenal tidak hanya karena keindahan alamnya, tetapi juga karena deretan rumah tua yang menghiasi kawasan tersebut. Menurut \cite{iqbal2020}, rumah-rumah tua ini memiliki keunikan dalam bentuk fisik, budaya, dan sejarahnya, menjadikannya elemen penting dalam memahami identitas dan dinamika ruang publik di daerah tersebut.
Namun, beberapa rumah tersebut telah diubah menjadi warung makan, dan jarak antara titik kumpul tepi laut dan bangunan yang berjauhan mengurangi makna dari bangunan tua tersebut.

Keakraban pengguna lokal dengan tepi laut Matras berkontribusi pada perasaan nyaman mereka di area tersebut.
Hal ini menunjukkan bahwa elemen fisik yang telah ada dalam jangka waktu yang lama dapat membentuk identitas suatu lokasi.
Kenyamanan tersebut tercermin dari lamanya waktu yang dihabiskan pengguna di area tersebut, terutama terlihat pada malam hari ketika tingkat keramaian meningkat.

Makna yang terletak pada bangunan lama sama pentingnya dengan makna yang baru saja terbentuk. Tepi laut Mattirotasi baru saja membangun sebuah pantai sebagai tempat rekreasi utama. Keutamaan ini dikarenakan tempat ini dapat menjadi lanmark yang memberikan makna pada sekitar tepi laut.
Sebagaiamana, makna dapat mengintegrasikan kebutuhan pengguna terkait aktivitas bersantai dengan keunikan lokasi pantai yang eksotis.

Hingga saat ini, masyarakat telah menikmati fungsi dari pantai ini. Itu tercerminkn oleh banyaknya aktivitas yang terjadi pada hari libur. Sebagai contoh, aktivitas \textit{speedboad} atau \textit{banana boat}, aktivitas ekonomi oleh UMKM (Usaha Mikro, Kecil, dan Menengah) dan aktivitas bersantai di pasir pantai.
Namun, pantai ini masih kurang dalam pemaknaan budaya dan lokalitas. Pada beberapa pantai di sekitar atau diluar objek penelitian, terdapat sejumlah objek yang mengisyaratkan budaya dan lokalitaas seperti contoh miniatur Kapal Layar Pinisi dan warna oranye sebagai bentuk lokalitas kota Parepare.

Penambahan elemen fisik dapat memperkuat makna yang mewakili kota Parepare \citep{atika2022}. Seperti contoh bangku, lampu, tempat sampah, desain rambu-rambu diwarnai warna oranye dan didesain bernuansa lokalitas. Hal ini dimaksudkan agar kawasan tepi laut ini memberikan makna yang sama bahwa pantai ini adalah bagian dari kota Parepare.



\section{Kesimpulan}%
\label{sec:Kesimpulan}

Tepi laut Matras dianalisis untuk menentukan bentuk fisiknya yang penting yang telah berkontribusi pada aspek-aspek ruang publik. Penelitian ini menyimpulkan terdapat sejumlah aspek yang perlu ditingkatkan dan dipertahankan pada setiap karakter tepi laut (bentuk fisik, aktivitas, dan makna) dalam memperkuat identitas ruang publik. Pada karakter bentuk fisik, aksesibilitas dan keterlihatan terhadap garis pantai dapat mengurangi kualitas identitas tepi laut. Masih banyak bentuk fisik yang tidak mendukung aspek aksesibilitas dan keterlihatan. Dalam hal aktivitas, ritel menjadi faktor pendorong utama dan memperkuat aktivitas yang terjadi di sekitar garis pantai. Pada karakter makna dan relevansi, rumah tua yang berjejeran sepanjang jalan raya tepi laut memberikan makna dan simbolisme. Namun, sedikit demi sedikit tempat ini beralih fungsi menjadi warung makanan. Setiap aspek mendukung karakter tempat dalam memperkuat identitas tepi laut yang unik. Aspek ini membantu memberikan pemahaman yang lebih baik tentang identitas tepi laut secara umum kepada pemangku kebijakan dan perancang perkotaan. Adapun pembahasan pada penelitian ini masih berdasarkan pada studi pilot, sehingga perlu dilakukan analisis yang lebih mendalam.




%----------------------------------------------------------------------------------------
%	BIBLIOGRAPHY
%----------------------------------------------------------------------------------------
% Comment these out if they are already in sub
\bibliographystyle{apalike}

% \bibliography{biblio.bib}
\bibliography{urbanidentity.bib}


\end{document}
